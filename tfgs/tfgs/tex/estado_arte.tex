% Contenidos del capítulo.
% Las secciones presentadas son orientativas y no representan
% necesariamente la organización que debe tener este capítulo.

\section{Análisis de aplicaciones similares}
% Qué aplicaciones similares hay y en qué se diferencia de ellas la propuesta
Como el TFG no se trata de una aplicación per se, si no de un asset destinado a el Asset Store de Unity, se puede determinar que no hay aplicaciones similares en el mercado tecnológico. No obstante, sí se pueden encontrar veintinueve Assets de teclados para realidad virtual en total, que oscilan entre la gratuidad y precios de hasta 359.27€. 

Los teclados de gama baja parecen muy simples respecto a uso, funciones y customización. Los de gama media son muy personalizables. En cuanto a la gama alta de teclados, destaca su robustez y capacidad para usos múltiples en todo el sistema. 

Para distinguirme de todos estos, decidí diseñar un teclado con el que se pueda escribir en varios idiomas, ya que ninguno de los veintinueve teclados antes mencionados poseen esta característica, acortando su rango de mercado, puesto que se limitan a la lengua inglesa.

\section{Tecnologías}
% Análisis crítico de las tecnologías y sistemas de despliegue posibles y por qué se han seleccionado unas concretas.
Para el desarrollo del teclado multilingüe en realidad virtual, se han considerado diversas tecnologías y sistemas de despliegue. A continuación, se presentan las opciones analizadas y las razones detrás de la selección final.

\textbf{Tecnologías de Desarrollo}

\textbf{-Unity:} Unity se seleccionó como la plataforma principal de desarrollo debido a su flexibilidad y robusto soporte para aplicaciones de RV. Su capacidad para integrarse con pluguins como TextMeshPro y mi entendimiento de esta por el grado estudiado fueron factores determinantes. Además permite el desarrollo en diversas plataformas, con lo que se alcanza una mayor audiencia, sumandose al factor de los múltiples idiomas.

\textbf{-Unreal Engine:} Aunque Unreal Engine ofrece gráficos de alta calidad y herramientas avanzadas para el desarrollo de RV, su curva de aprendizaje es más pronunciada y dado a que debería de haber aprendido de cero, el tiempo requerido para la realización del proyecto se habría prolongado excesivamente, volviéndose inviable para terminar en el plazo requerido. Dado el enfoque del proyecto en la usabilidad y la eficiencia del desarrollo, Unity fue la opción preferida. Además, la gestión de recursos y la documentación extensa en Unity facilitaron la implementación del teclado multilingüe.

\textbf{Sistemas de despliegue}

\textbf{-Asset Store:} Asset store se seleccionó como uno de los sistemas de despliegue por distintos motivos. Es una plataforma muy conocida y ampliamente utilizada por los desarrolladores, lo que brinda una altísima exposición. El hecho de que el asset esté disponible en el Asset Store de Unity puede servir como una validación de su calidad y utilidad. Los usuarios pueden confiar en que los assets han pasado por un proceso de revisión y cumplen con los estándares de calidad de Unity, lo que puede aumentar la credibilidad y la confianza en el producto. Es muy fácil descubrir, adquirir y administrar los distintos assets ya que utiliza la infraestructura existente de Unity para la distribución y gestión de estos. La publicación aquí te permite actualizar regularmente los assets y ofrecer soporte a los usuarios de manera eficiente. Los usuarios pueden recibir notificaciones automáticas de actualizaciones y acceder fácilmente a la documentación y el soporte técnico que proporciones a través de la plataforma. Y por último, pero no menos importante, ofrece la posibilidad de monetizar tu trabajo vendiendo tu asset a otros desarrolladores,  que es parte del objetivo de este proyecto.

\textbf{-GitHub:} Aunque GitHub también es considerablemente conocido por los desarrolladores, la integración con GitHub Actions para la automatización de flujos de trabajo de CI/CD ayuda muchísimo a la hora de ejecutar pruebas y el control de versiones, no obstante el hecho de que tendría que aprender de cero el funcionamiento de GitHub Actions y mi objetivo de comercializar el proyecto, me hacen poco atractiva esta idea. Dicho esto, sigo utilizándolo para guardar el proyecto para mi uso personal, no elimino la idea de utilizarlo para venderme como programador.

